\documentclass[
        %handout,
        %draft,
        ]{beamer}
\usepackage{amssymb,latexsym,amssymb,amsmath,amsbsy,amsopn,amstext,upgreek}
\usepackage{color,multicol}
\usepackage{graphicx,wrapfig,fancybox,watermark,graphics}
\usepackage{picins}
%\usepackage{emp}
%\usepackage{pstricks,pst-plot}
\usepackage{pgf}
\usepackage{movie15}
\usepackage{hyperref}
\usepackage{pdfpages}
\usepackage{listings,bera}
\definecolor{keywords}{RGB}{255,0,90}
\definecolor{comments}{RGB}{60,179,113}
\lstset{language=C,
keywordstyle=\color{keywords},
commentstyle=\color{comments}\emph}
\hypersetup{
    pdfpagemode=FullScreen, % show in full screen?
}
\usepackage{algorithm}
\usepackage{algorithmic}
\renewcommand{\algorithmicrequire}{\textbf{Input:}}
\renewcommand{\algorithmicensure}{\textbf{Output:}}
% reference entry
\usepackage{bibentry, natbib}
% reference style
\bibliographystyle{IEEEtran} 
%reference lib
\nobibliography{refs}

\usepackage[
	%compress,
	%minimal,
	%nonav,
	red,
	%gold,
	%numbers,
	%nologo,
	polyu,
	]{beamerthemeHongKong}
\usefonttheme[professionalfonts]{serif}

\title[Tutorial 11]{Tutorial 11: Assignment 5}
\author[COMP210]{Qu Xiaofeng\texorpdfstring{, Teaching Asistant\\\tiny{quxiaofeng.at.polyu@gmail.com, PQ702}}{}}
\institute{COMP210\\Discrete Structure}
\date{\today}

\begin{document}

\frame{\titlepage}

\section*{Table of Contents}

    \begin{frame}[shrink]{\secname}
        \tableofcontents
    \end{frame}

\AtBeginSubsection[] {
    \begin{frame}<handout:0>[shrink]{Outline}
        \tableofcontents[current,currentsubsection]
    \end{frame}
}



\section{Problems}
    \subsection{Problem 1}    
        \begin{frame}[c]{\subsecname}
            \emph{In how many ways can five distinct Martians and five distinct Jovians be seated at circular table?}\\$\;$\\\pause
            The answer is 9!. \qed
        \end{frame}



    \subsection{Problem 3}
        \begin{frame}[c]{\subsecname}
            \emph{Find the number of integer solutions of $x_1 + x_2 + x_3 = 15$ subject to $ x_1, x_2, x_3 \geq 0$.}\\$\;$\\\pause
            The answer is C(15+3-1,15). \qed
        \end{frame}



    \subsection{Problem 6.1}
        \begin{frame}[c]{\subsecname}
            \emph{Four microprocessors are randomly selected from 100 microprocessors among which 10 are defective. Find the probability of obtaining no defective microprocessors?} \\$\;$\\ \pause
            The answer is C(90,4)/C(100,4). \qed
        \end{frame}



    \subsection{Problem 7.1}
        \begin{frame}[c]{\subsecname}
            \emph{Consider three persons who each randomly choose a locker among 12 consecutive
lockers. What is the probability that the three lockers chosen are consecutive.}\\$\;$\\\pause
            The answer is 10/C(12,3). \qed
        \end{frame}


\section{Problems cont.}
    \subsection{Problem 10.1}
        \begin{frame}[c]{\subsecname}
            \begin{columns}
            \column{0.75\textwidth}
                \emph{State which graphs are bipartite graphs. If the graph is bipartite, specify the disjoint vertex sets}\\$\;$\\
                \uncover<2->{The answer is Bipartite. $V_1=\{v_1,v_2,v_5\},\; V_2=\{v_3,v_4\}$. \qed}
            \column{28.6mm}
                \begin{figure}
                    \centering
                    \includegraphics[width=28.6mm]{tut11p10}
                \end{figure}
            \end{columns}
        \end{frame}
        
        

    \subsection{Problem 13.1}
        \begin{frame}[c]{\subsecname}
            \begin{columns}
            \column{0.75\textwidth}
                \emph{Consider the graph in the figure. Find the degree for each vertex in the graph}\\$\;$\\
                \uncover<2->{The answer is that every vertex has degree 4. \qed}
            \column{34mm}
                \begin{figure}
                    \centering
                    \includegraphics[width=34mm]{tut11p13_1}
                \end{figure}
            \end{columns}
        \end{frame}



    \subsection{Problem 14}
        \begin{frame}[c]{\subsecname}
            \emph{Show that the maximum number of edges in a simple, bipartite graph with n
vertices is (n-1)(n-2)/2.}\\$\;$\\\pause
            Let $G$ be a simple, disconnected graph with $n$ vertices having the maximum number of edges. Show that G has two components. If one component has $i$ vertices, show that the components are $K_i$ and $K_{n-i}$. Use Excercise 11, Section 8.1, to find a formula for the number of edges in $G$ as a function of $i$. Show that the maximum occurs when $i=1$. \qed
        \end{frame}



    \subsection{Problem 17.1}
        \begin{frame}[c]{\subsecname}
            \begin{figure}
                \centering
                \includegraphics[width=96.3mm]{tut11p17_1}
            \end{figure}
            \emph{Find the length of a shortest path and a shortest path between each pair of
vertices in the weighted graph. 1) a,f.}\\$\;$\\\pause
            The answer is 7;(a, b, c, f). \qed
        \end{frame}


\section{Problems cont.}
    \subsection{Problem 19}
        \begin{frame}[c]{\subsecname}
            \emph{Show that a tree is a bipartite graph.}\\$\;$\\\pause
            Let $T$ be a tree. Root $T$ at some arbitrary vertex. Let $V$ be the set of vertices on even levels and let $W$ be the set of vertices on odd levels. Since each edge is incident on a vertex in $V$ and a vertex in $W$, $T$ is a bipartite graph.\qed
        \end{frame}



    \subsection{Problem 20}
        \begin{frame}[c]{\subsecname}
            \emph{Show that a graph $G$ with $n$ vertices and fewer than $n-1$ edges is not connected.}\\$\;$\\\pause
            Suppose that $G$ is connected. Add parallel edges until the resulting graph $G^*$ has $n-1$ edges. Since $G^*$ is connected and has $n-1$ edges, by Theorem 9.2.3, $G^*$ is acyclic. But adding an edge in parallel introduces a cycle. Contradiction. \qed
        \end{frame}



    \subsection{Problem 23.1}
        \begin{frame}[c]{\subsecname}
            \emph{Find a spanning tree for each graph using depth-first and breadth-first search.}\\
            \begin{figure}
                \centering
                \includegraphics[width=99.6mm]{tut11p23_1_1}
            \end{figure}\pause
            \begin{figure}
                \centering
                \includegraphics[width=93mm]{tut11p23_1_2} \qed
            \end{figure}
        \end{frame}



    \subsection{Problem 24}
        \begin{frame}[c]{\subsecname}
            \emph{Let $T$ and $T^\prime$ be two spanning trees of a connected graph $G$. Suppose that an edge $x$ is in $T$ but not in $T^\prime$. Show that there is an edge $y$ in $T^\prime$ but not in $T$ such that $( T - \{ x \} ) \cup \{ y \} $ and $(T^\prime - \{ y \} ) \cup \{ x \} $ are spanning trees of $G$.}\\$\;$\\\pause
            Suppose that $x$ is incident on vertices $a$ and $b$. Removing $x$ form $T$ produces a disconnected graph with two components, $U$ and $V$. Vertices $a$ and $b$ belong to different components\--- say, $a \in U$ and $b \in V$. There is a path $P$ from $a$ to $b$ in $T^\prime$. As we move along $P$, at some point we encounter an edge $y=(v,w)$ with $v \in U$, $w \in V$. Since adding $y$ to $T-\{x\}$ produces a connected graph, $(T-\{x\}) \cup \{y\}$ is a spanning tree. Clearly, $(T^\prime-\{y\}) \cup \{x\}$ is a spanning tree. \qed
        \end{frame}
    
    
    
    \begin{frame}<handout:0>[c]{\secname}
        \centerline{\Large{Questions about the problems?}}
    \end{frame}
    
    
    
\end{document}



